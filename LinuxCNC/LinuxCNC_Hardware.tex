% Porter model
% Author: Charles-Axel Dein
% \documentclass[10pt,a3paper]{article} 
\documentclass[10pt,a4paper]{article} 

\usepackage[hmargin=2cm,vmargin=1cm]{geometry}
\renewcommand{\rmdefault}{bch} % change default font

\usepackage[english]{babel}
\usepackage[utf8]{inputenc}
\usepackage{tikz} 
\usetikzlibrary{arrows,decorations.pathmorphing,backgrounds,fit,positioning,shapes.symbols,chains}

\usepackage[active,tightpage,floats]{preview}
\setlength\PreviewBorder{5pt}%

%%%%%%%%%%%%%%%%%%%%%%%%%%%%%%%%%%%%
%%% BEGIN DOCUMENT
\begin{document}

\begin{figure}[h]

\centering
\begin{tikzpicture}
[node distance = 1cm, auto,font=\footnotesize,
% STYLES
every node/.style={node distance=3cm},
% The comment style is used to describe the characteristics of each force
comment/.style={rectangle, inner sep= 5pt, text width=4cm, node distance=0.25cm, font=\scriptsize\sffamily},
% The force style is used to draw the forces' name
force/.style={rectangle, draw, fill=black!10, inner sep=5pt, text width=4cm, text badly centered, minimum height=1.2cm, font=\bfseries\footnotesize\sffamily}] 

% Draw forces
\node [force] (HEAD) {LINUXCNC\_PC};
\node [force, below of=HEAD] (PCI) {PCI CARD/PCIe CARD};
%\node [force, right=5cm of=PCI] (EtherCAT) {EtherCAT};
\node[force, right=1cm of PCI] (EtherCAT) {EtherCAT};
\node[force, right=1cm of EtherCAT] (FPGA) {I/O FPGA Cards};



%\node [force] (HEAD) {PCI CARD/PCIe CARD};
%\node [force, above of=HEAD] (substitutes) {LINUXCNC PC 1};

%\node[draw=blue, fill=yellow!20, thick, rounded corners, text width=3cm, align=center, below=of X]
% (myNode) {Example node\\with styling};

%\node [draw=blue, fill=yellow!20, thick, rounded corners, text width=3cm, align=center, below=of  substitutes] (state) {draw=blue, fill=yellow!20, thick, rounded corners, text width=3cm, align=center, below=of  substitutes};
%\node [force, text width=3cm, dashed, left=1cm of substitutes] (state) {Public policies};
% \node [force, left=1cm of HEAD] (suppliers) {Bargaining power of suppliers};
% \node [force, right=1cm of HEAD] (users) {Bargaining power of users};
%\node [force, below of=HEAD] (entrants) {LINUXCNC PC};



% | Option              | Description                   | Example               |
% | ------------------- | ----------------------------- | --------------------- |
% | `text width=3cm`    | Wraps text to specified width | `text width=2cm`      |
% | `align=center`      | Centers multiline text        | `align=center`        |
% | `dashed`            | Makes border dashed           | `dashed`              |
% | `dotted`            | Dotted border                 | `dotted`              |
% | `thick`             | Thicker border                | `thick`               |
% | `draw=red`          | Border color                  | `draw=blue`           |
% | `fill=yellow`       | Background color              | `fill=gray!20`        |
% | `rounded corners`   | Rounded node corners          | `rounded corners=3pt` |
% | `minimum width=3cm` | Force fixed width             | `minimum width=2cm`   |
% | `font=\small`       | Font size/style               | `font=\itshape`       |
% | `inner sep=5pt`     | Padding inside node           | `inner sep=2pt`       |
% 




%%%%%%%%%%%%%%%
% Change data from here



% PCI
% \node [comment, right=0.25cm of PCI] {
% (+) PCI (Peripheral Component Interconnect): Older standard, slower data transfer speeds (up to 133 MB/s).\\
% (+) PCIe (PCI Express): Newer, faster standard with data rates up to 32 GB/s (for PCIe 5.0).\\
% (+) Form Factor: PCI uses 32-bit or 64-bit slots, while PCIe uses lane-based architecture (x1, x4, x8, x16).\\
% (+) PCI is an older parallel interface; PCIe (PCI Express) is a newer high-speed serial interface.\\
% (+) PCIe offers higher bandwidth and scalability via lanes (x1, x4, x8, x16), unlike fixed PCI.\\
% (+) PCIe slots are physically different and not backward compatible with PCI cards.};



% FPGA
\node [comment, below=0.25cm of FPGA] {
(\*)https://github.com/LinuxCNC/mesaflash};

% \node [comment, below=0.25 of HEAD] (comment-HEAD) {(+) A war against Microsoft\\
% (+) Limiting sunk costs\\
% (+) Coopetition};





% SUBSTITUTES
% \node [comment, right=0.25 of HEAD] {(+) Portability};
% \node [comment, left=0.25 of HEAD] {(+) Portability};


% USERS
%\node [comment, below=0.25 of users] {(+) Increasing the user information\\
%(+) Reducing the switching costs};
%
% NEW ENTRANTS
% \node [comment, right=0.25 of entrants] {(+) EC vs. Microsoft};


%	\node[left=1cm of A]    (B) {Text};  % Left of node A
%	\node[right=1cm of A]   (C) {Text};  % Right of node A
%	\node[below=1cm of A]   (D) {Text};  % Below node A
%	\node[above=1cm of A]   (E) {Text};  % Above node A





% PUBLIC POLICIES
%	\node [comment, text width=3cm, below=0.25 of state] {(+) Positively framed\\
%	(+) Transparency\\
%	(--) A new monopoly?};





%%%%%%%%%%%%%%%%

% Draw the links between forces
% \path[->,thick]  (PCI) edge (HEAD);
% \path[->,thick, bend left]  (HEAD) edge (PCI);    % Curved left
% \path[->,thick, bend right] (PCI) edge (HEAD);    % Curved right
% \path[->,thick, out=45, in=135] (PCI) edge (HEAD);  % Custom curve
% \path[<->, thick] (HEAD) edge (PCI);  % Arrow both directions
% \path[->, thick, dashed] (HEAD) edge (PCI);     % Dashed arrow
\path[->, ultra thick, dotted] (HEAD) edge (PCI);
\path[->, ultra thick, dotted] (HEAD) edge (EtherCAT);
\path[->, ultra thick, dotted] (HEAD) edge (FPGA);

% (suppliers) edge (HEAD)
% (users) edge (HEAD)
%(entrants) edge (comment-HEAD);

\end{tikzpicture} 
\caption{LinuxCNC Hardware}
\label{fig:6forces}
\end{figure}

\end{document}